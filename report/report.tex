\documentclass[conference]{IEEEtran}
\IEEEoverridecommandlockouts

\usepackage{cite}
\usepackage{amsmath,amssymb,amsfonts}
\usepackage{algorithmic}
\usepackage{graphicx}
\usepackage{textcomp}
\usepackage{xcolor}

\def\BibTeX{{\rm B\kern-.05em{\sc i\kern-.025em b}\kern-.08em
    T\kern-.1667em\lower.7ex\hbox{E}\kern-.125emX}}

\title{Personalized Vegan Food Recommender System Using a Hybrid Machine Learning Approach}

\author{
\IEEEauthorblockN{Zuriel Aviles}
\IEEEauthorblockA{
\textit{B.S. Computer Science, Cognitive Science} \\
\textit{Rensselaer Polytechnic Institute}\\
Troy, New York \\
avilez@rpi.edu}
}

\begin{document}

\maketitle

\begin{abstract}
With the growing interest in plant-based diets, there's an increasing need for tools that support personalized food recommendations. This paper presents a hybrid vegan food recommender system leveraging both content-based and collaborative filtering, trained on open recipe datasets. We demonstrate the effectiveness of integrating user preferences, nutritional information, and sentiment from user reviews using machine learning models.
\end{abstract}

\begin{IEEEkeywords}
Recommender Systems, Content-Based Filtering, Collaborative Filtering, Vegan, Machine Learning, Nutrition, NLP
\end{IEEEkeywords}

\section{Introduction}
Explain the rise of veganism, the information overload problem in recipe platforms, and the limitations of existing food recommendation systems. Introduce your goal of building a personalized recommender.

\section{Related Work}
Summarize key works (e.g., from NYC Data Science Capstone, RecSys literature, and hybrid recommendation techniques). Highlight the gap your system fills.

\section{Dataset and Preprocessing}
Detail the datasets used: e.g., \textit{Food.com}, \textit{RecipeNLG}, Spoonacular API. Discuss how you filtered vegan recipes and engineered features (ingredients, nutrition, user reviews).

\section{Methodology}
\subsection{Content-Based Filtering}
Explain ingredient vectorization (TF-IDF, embeddings) and similarity metrics.

\subsection{Collaborative Filtering}
Mention matrix factorization or k-NN methods based on user-item interactions.

\subsection{Hybrid Approach}
Describe how you combine the above (e.g., weighted average, model stacking).

\subsection{Incorporating NLP and Sentiment}
Explain review processing using NLP, extracting sentiment to refine recommendations.

\section{Model Training and Evaluation}
Detail model training procedures, data splits, and evaluation metrics (precision, recall, F1, user satisfaction if applicable).

\section{Results and Discussion}
Present results with tables/graphs. Analyze which models performed best and why.

\section{Conclusion and Future Work}
Summarize contributions. Suggest extensions such as deep learning, real-time feedback, voice interface, or health goal optimization.

\section*{Acknowledgment}
Optionally acknowledge your advisor, dataset providers, or any funding.

\begin{thebibliography}{00}
\bibitem{b1} NYC Data Science Academy, "Recipe Recommendation Using NLP and Clustering," https://nycdatascience.com/blog/student-works/capstone/recipe-recommendation/
\bibitem{b2} Other references from your proposal or citation list.
\end{thebibliography}

\end{document}
